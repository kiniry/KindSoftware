%
% $Id: lajug_8_aug_2002.tex 7739 2008-02-15 17:08:37Z dcochran $
%

\documentclass[%
%draft,
final,
slideColor,
%slideBW,
%total,
nototal,
nocolorBG,
%colorBG,
%ps,
pdf,
accumulate,
%noaccumulate,
%distiller,
next,
%alcatel,
%alienglow,
%autumn,
%azure,
%contemporain,
%darkblue,
%frames,
%gyom,
%lignesbleues,
%nuancegris,
%pascal,
%rico,
%troispoints,
]{prosper}

%=====================================================================

\usepackage{graphicx}
\usepackage{prs}
\usepackage{warn}
%\usepackage{html}
\usepackage{url}
\usepackage{alltt}
\usepackage{xspace}
\usepackage{epsfig}
\usepackage{changebar}
\usepackage{amsmath}
\usepackage{amsfonts}
%\usepackage{amsthm}
%\usepackage{amscd}
\usepackage{amssymb}
\usepackage{eucal}
%\usepackage{eufrak}
%\usepackage{oz}
\usepackage[centredisplay,PostScript=dvips]{diagrams}
\usepackage{float}
%\usepackage[dvips]{hyperref}
\usepackage{trig}
%\usepackage[graphics]{ite}
%\usepackage{ite}

% Lucida Bright Fonts
%\usepackage[LY1]{fontenc}
%\usepackage[expert,LY1]{lucidabr}

%=====================================================================
% New commands, macros, etc.
%=====================================================================

\include{newcommands}

%=====================================================================
% Title page
%=====================================================================

\title{Using Java for VLSI Development\\
\begin{small}and/or\end{small}\\
Advanced Open Source Java Software Engineering}
\author{Joseph R.~Kiniry}
\institution{KindSoftware, LLC and Fulcrum Microsystems, Inc.}
\slideCaption{LAJUG --- August 2002}
%\Logo(-1,-1){\includegraphics[width=1cm]{cit_gray_logo.eps}}
%\displayVersion
%\DefaultTransition{Replace}

%=====================================================================
% Beginning
%=====================================================================

\begin{document}
\maketitle
%\special{papersize=297mm,210mm}

%=====================================================================
% Template
%=====================================================================

%\begin{slide}{Title}
%  \begin{itemize}
%    \item
%  \end{itemize}

%  \begin{small}
%  \end{small}

%  \begin{equation*}
%    \begin{array}{ccc}
%    \end{array}{ccc}
%  \end{equation*}

%  \begin{diagram}[textflow]
%  \end{diagram}

%  \includegraphics[width=8cm]{.eps}
%\end{slide}

%=====================================================================
% Slides
%=====================================================================

\begin{slide}{Java for AVLSI?}
  \begin{itemize}
  \item What is AVLSI?
    \begin{itemize}
    \item Delay insensitive circuits
    \item Power invariant
    \item Design scalability
    \item Process invariant
    \end{itemize}
  \item How is VLSI design typically done?
    \begin{itemize}
    \item High level specification (e.g., VHDL)
    \item Low-level specification (e.g., Verilog)
    \item Automated layout
    \end{itemize}
  \item How are we doing it?
  \item How is Java used?
  \end{itemize}
\end{slide}

\begin{slide}{Challenges}
  \begin{itemize}
  \item Performance
    \begin{itemize}
    \item You try simulating a CPU in Java!
    \end{itemize}
  \item Scalability
    \begin{itemize}
    \item Massive memory and thread use
    \end{itemize}
  \item Robustness
    \begin{itemize}
    \item If simulation takes five days and it crashes on day four...
    \end{itemize}
  \item \textbf{Correctness!}
    \begin{itemize}
    \item Fabricating a chip = no patches
    \end{itemize}
  \end{itemize}
\end{slide}

\begin{slide}{Design Process}
  \begin{itemize}
  \item Multiple specification levels.
    \begin{itemize}
    \item Multiple Java realizations
    \item CSP (Concurrent Sequential Processes)
    \item Production rules
    \item Layout
    \end{itemize}
  \item Unit testing
    \begin{itemize}
    \item Cell, Unit, CPU
    \item With and without OS
    \item At multiple refinement levels
    \end{itemize}
  \item Cosimulation for behavioral equivalence
    \begin{itemize}
    \item Formal refinement checking
    \end{itemize}
  \end{itemize}
\end{slide}

\begin{slide}{Observations}
  \begin{itemize}
  \item (Mis)use of concurrency
    \begin{itemize}
    \item Thread per anything
    \end{itemize}
  \item Data structure (ab)use
    \begin{itemize}
    \item Generic Java data structures
    \end{itemize}
  \item Aimless optimization
    \begin{itemize}
    \item (Aimless) optimization is the root of all evil.
    \end{itemize}
  \item Untracked requirement changes
    \begin{itemize}
    \item Complexity has a requires clause
    \end{itemize}
  \item Documentation process
    \begin{itemize}
    \item Least favorite part coupled with rapid corporate development and hard
    delivery dates.
    \end{itemize}
  \end{itemize}
\end{slide}

\begin{slide}{Response}
  \begin{itemize}
  \item Commercial tools where necessary
    \begin{itemize}
    \item Analysis: JProbe and JProfiler
    \item Revision control: P4 (was CVS)
    \item Simulation: Cadence
    \end{itemize}
  \item Open Source tools where possible
    \begin{itemize}
    \item Custom code coverage: Gretel
    \item Metrics: Java NCSS, SlocCount
    \item Documentation: SGML and \LaTeX\
    \item Specification: JML
    \item Build system: Ant
    \end{itemize}
  \item Process, process, process
    \begin{itemize}
    \item Documentation
    \item \textbf{Specification}
    \end{itemize}
  \end{itemize}
\end{slide}

\begin{slide}{Results}
  \begin{itemize}
  \item Performance
    \begin{itemize}
    \item Typical: 10 minute change = 10 percent
    \item Atypical: one man month = 1000 percent
    \item ...and nothing in between.
    \end{itemize}
  \item Memory use
    \begin{itemize}
    \item Garbage collection (ab)use
      \begin{itemize}
      \item Iterators, Events, and StringBuffers
      \end{itemize}
    \item OS VM abuse
    \item Overall memory size
    \end{itemize}
  \item System montoring
    \begin{itemize}
    \item Subsystem tailored to design space
    \item Optional compilation
    \item Framework licensing (IDebug)
    \end{itemize}
  \end{itemize}
\end{slide}

\begin{slide}{Key Aspects}
  \begin{itemize}
  \item Lightweight specification
    \begin{itemize}
    \item Semantic properties via Javadoc
    \item Standard system overviews
      \begin{itemize}
      \item Abstract
      \item Overview
      \item Requirements
      \item Dictionary
      \end{itemize}
    \item System tracking
      \begin{itemize}
      \item Development state
      \item Deliverabes
      \item Tasks
      \item Risk analysis
      \end{itemize}
    \end{itemize}
  \item High-level specification
    \begin{itemize}
    \item Extended BON
    \end{itemize}
  \end{itemize}
\end{slide}

\begin{slide}{Key Aspects, cont.}
  \begin{itemize}
  \item Detailed specification
  \item Several alternatives, OS and commerical
    \begin{itemize}
    \item iContract, Jass, JML, MetaMata, jContract
    \end{itemize}
  \item Design by Contract as a start
  \item Model-based specification for completeness
  \item JML = Java Modeling Language
  \end{itemize}
\end{slide}

\begin{slide}{Semantic Properties}
  \begin{itemize}
  \item Domain-specific specification constructs that augment an existing
    language with richer semantics.
  \item Used in system analysis, design, implementation, testing, and
    maintenance via documentation and source-code analysis and
    transformation tools.
  \item Three forms: \emph{informal}, \emph{semi-formal}, and
    \emph{formal}.
  \item Advance over existing work because:
    \begin{itemize}
    \item Specified as annotations, which is popular with programmers.
    \item More natural model because little-to-no math.
    \item Higher-level contructs, thus more expressive.
    \end{itemize}
  \end{itemize}
\end{slide}

\begin{slide}{Example of Annotated Java}
\tiny
\begin{verbatim}
/**
 * Returns a boolean indicating whether any debugging facilities 
 * are turned off for a particular thread.
 *
 * @concurrency GUARDED
 * @require (thread != null) Parameters must be valid.
 * @modify QUERY
 * @param thread the thread to check.
 * @return a boolean indicating whether any debugging facilities 
 * are turned off for the specified thread.
 * @review kiniry Are the isOff() methods necessary at all?
 **/
     
 public synchronized boolean isOff(Thread thread)
 {
   return (!isOn(thread));
 }
\end{verbatim}
\end{slide}

\begin{slide}{Extended BON, cont.}
  \begin{small}
    \begin{itemize}
    \item Extensions - semantically well-understood best practice
      programming constructs in specific domains.
      \begin{itemize}
      \item \emph{concurrency} - sequential, concurrent, guarded.
      \item \emph{temporal logic contracts} - for run-time testing.
      \item \emph{modifies} - predicates about state change.
      \item \emph{generates} - concurrency dynamism.
      \end{itemize}
    \item Bijective refinement.
      \begin{itemize}
      \item specification changes $\leftrightarrow$ program changes.
      \end{itemize}
    \item ``Weak'' language plus quality tools $\Rightarrow$ \newline
      stronger system than most of industry's best practices.
    \end{itemize}
  \end{small}
\end{slide}

\begin{slide}{Extended BON}
  \begin{footnotesize}
    \begin{itemize}
    \item A lightweight specification language for software.
    \item A design model checker for multiple languages.
    \item Extended BON = BON + semantic properties.
    \item BON - the Business Object Notation.
      \begin{itemize}
      \item Textual and graphical syntax.
      \item Well-understood semantics.
      \item Seamless, reversible specification language.
      \end{itemize}
    \item BON specifications capture important system aspects...
      \begin{itemize}
      \item static and dynamic structure, contracts, test cases.
      \end{itemize}
    \item ...but are not complete and are relatively weak.
    \end{itemize}
  \end{footnotesize}
\end{slide}

\begin{slide}{JML}
  \begin{itemize}
  \item Syntax is extension of Java
  \item Contracts a la Eiffel
    \begin{itemize}
    \item requires, ensures and interplay with inheritance
    \end{itemize}
  \item Side-effects (or lack thereof)
    \begin{itemize}
    \item pure, assignable
    \end{itemize}
  \item Refinement (spec to impl, spec to spec)
    \begin{itemize}
    \item also, refines
    \end{itemize}
  \item Specification scoping (contract visibility)
  \item Model variables
    \begin{itemize}
    \item Software cosimulation
    \end{itemize}
  \item Documentation generation, static analysis, and dynamic run-time checks
  \end{itemize}
\end{slide}

\begin{slide}{JML Tools}
  \begin{itemize}
  \item JMLDoc
    \begin{itemize}
    \item JML plus Javadoc
    \end{itemize}
  \item Typechecking
    \begin{itemize}
    \item jml
    \end{itemize}
  \item Compilation
    \begin{itemize}
    \item jmlc
    \end{itemize}
  \item Static analysis
    \begin{itemize}
    \item ESCJava, LOOP, Daikon, etc.
    \end{itemize}
  \end{itemize}
\end{slide}

\begin{slide}{For More Information}
  \begin{itemize}
  \item JML
    \begin{itemize}
    \item \texttt{http://www.jmlspecs.org/}
    \end{itemize}
  \item Extended BON
    \begin{itemize}
    \item \texttt{http://ebon.sf.net/}
    \end{itemize}
  \item Fulcrum Microsystems
    \begin{itemize}
    \item \texttt{http://www.fulcrummicro.com/}
    \end{itemize}
  \item KindSoftware
    \begin{itemize}
    \item \texttt{http://www.kindsoftware.com/}
      \begin{itemize}
      \item IDebug, semantic properties, code standards, formal methods and
      advanced software engineering consulting, etc.
      \end{itemize}
    \end{itemize}
  \end{itemize}
\end{slide}

%% \begin{slide}{Title}
%%   \begin{itemize}
%%   \item
%%   \end{itemize}
%% \end{slide}

%% \begin{slide}{Title}
%%   \begin{itemize}
%%   \item
%%   \end{itemize}
%% \end{slide}

%% \begin{slide}{Title}
%%   \begin{itemize}
%%   \item
%%   \end{itemize}
%% \end{slide}

%% \begin{slide}{Title}
%%   \begin{itemize}
%%   \item
%%   \end{itemize}
%% \end{slide}

%% \begin{slide}{Title}
%%   \begin{itemize}
%%   \item
%%   \end{itemize}
%% \end{slide}

%---------------------------------------------------------------------
% End of Document
%---------------------------------------------------------------------

\end{document}

%%% Local Variables: 
%%% mode: latex
%%% TeX-master: t
%%% End: 

